\documentclass[11pt]{article}

\usepackage{amsmath, amssymb, amsthm}
\usepackage[all]{xy}
\usepackage{color}
\usepackage[pdftex]{graphicx}

\usepackage{fullpage}
%\usepackage[top=1.5in,bottom=1.5in,left=2in,right=2in]{geometry}

\begin{document}

\title{H2O IQ - Business Plan}
\author{Mark Fuge \and Shiry Ginosar \and Valkyrie Savage}

\maketitle

\section{Mission Statement}

H2O IQ allows gardeners to water their plants efficiently and consistently whether they are at home or on vacation, available or busy. The system allows gardeners to monitor the moisture intake of their plants, and set a drip irrigation system to automatically achieve the ideal conditions for each plant without over watering or losing water to evaporation. Thus, H2O IQ saves time, conserves water and prevents the guesswork involved with achieving perfect growing conditions.

\section{Price}

Based on existing products targeting the same market share, we project that H2O IQ designed for 5 plant groups will cost \$99 at initial retailers. 

\section{Cost}

\section{H2O IQ System Description}

H2O IQ is a system designed to monitor and actuate drip irrigation in a garden setting. It comprises of in-garden solar powered interactive wireless sensor/actuator devices, an in-garden wireless central server and an online interface. Gardeners can use H2O IQ in one of two modes. In a manual mode, gardeners water their garden themselves and only use the online interface in order to monitor the current moisture level of the plants in order to decide when to water next and how much. In this mode, gardeners can issue a one-time watering command from the online interface in case they are too busy to water on a specific day. In an automatic mode, gardeners set thresholds for an automatic watering regime that will water their plants each time the moisture level gets below the threshold.

Each interactive sensor device is designed to monitor a group of plants of the same species (for instance, one device could monitor a group of tomato plants). The pointy end of the device is intended to be placed in the ground next to one of the plants to be monitored. It consists of a solar powered moisture sensor, a drip irrigation valve actuator and input buttons.

The central in-garden server is a wireless server designed to control all sensor/actuator devices in one garden. This design reduces the cost of the wireless communication to each sensor, as a lower powered radio connection can be used within the garden. The server also acts as a web server that hosts the online interface.

\subsection{System Architechture}

\section{Intended Market}

Our target users are young professionals or graduate students, who maintain vegetable gardens at home or in community gardens. We chose to target this group of people since they are interested in gardening for reasons of food production, sustainability, eating locally etc. and enjoy gardening, but do not always have enough time to devote to their garden as may be required due to their busy lifestyle.

\subsection{Potential Use Case Scenario}
% Situation, Tasks, Users

\subsubsection{Key Task 1 - Watering on a regular basis}
Plants need water, and the main job of a gardener once the plot has been planted and weeded is to water the garden. Some gardeners do this every other day, some every week, but all have some regime that they follow and certain amounts of water that they give each plant on a regular basis. The main problem with watering is the time it takes to fulfill the task in an often busy lifestyle. Current solutions to this problem, like automatic sprinklers or drip irrigation take the responsibility for watering away from the gardener, who often enjoys tending to the garden when they do have time. Moreover, these solutions do not take into account the unique needs of each plant (see task 2).

H2O IQ has two modes. In the manual mode, the system monitors the moisture levels in each plants and notifies the gardener when it is time to water. In the automatic mode, the system waters the garden by actuating a drip irrigation according to the readings from the moisture sensors in the ground and the thresholds set by the user.

\subsubsection{Key Task 2 - Figuring out how much water to give to each plant}

Each species and plant has unique water needs that are highly dependent on the microclimate, the sun and the soil. Figuring out how much water to give each plant is hard, and is often done by observation of the plant status and by using tools like moisture meters. A common problem with moisture sensors and meters is that their readings change when the prongs move and when the ground is fertilized. As the meter or sensor only measures the moisture in the ground, it does not have a good reading of the extent to which plant roots can absorb the water from the ground.

Moreover, companion planting raises the need to give close by plants different amounts of water. Current solutions for close by plants include watering cones and drip irrigation that are delivered . However, both these solutions need to be configured by hand in a long trial and error process for each plant.

In H2O IQ's automatic mode, the drip irrigation regime is specialized for each plant group based on the reading of its sensors. In order to make more reliable sensors, we use pairs of galvanized nails packaged in plaster. The packaging keeps the distance between the nails constant, and the plaster absorbs water from the ground in a similar way to plant roots and is not affected by fertilization. Thus our sensor is measuring the moisture in the plant rather than the moisture in the ground.

\section{Manufacturing Costs}

\subsection{Most Likely Case Product Cost}
\subsection{Design and Development Costs}
\subsection{Tooling Costs}
\subsection{Materials Cost}
\subsection{Labor Cost}
\subsection{Production Costs}
\subsection{Overhead}

\section{Operational Costs}

\section{Most Likely Scenario for Profit Model Spreadsheets}

\section{The H2O IQ Corporation}

\subsection{Corporate Structure}
% Organizational chart goes here

\section{Competitive Landscape}

\section{Model Spreadsheets}

%\begin{figure}[h]
%\begin{center}
%\includegraphics[scale=0.36]{./pngs/XXX.png}
%\end{center}
%\caption{YYY}
%\label{fig:XXX}
%\end{figure}

\bibliographystyle{plain}
\bibliography{businessplan}
\end{document}